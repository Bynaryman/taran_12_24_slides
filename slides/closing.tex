\graphicspath{{./figs/}}
% References if we want
%\begin{frame}[allowframebreaks]{References}
%\printbibliography
%\end{frame}

\begin{frame}[plain]
    \frametitle{Agenda}

    \tableofcontents[currentsection, subsectionstyle=show/shaded/hide]

\end{frame}

\begin{frame}
    \frametitle{Conclusions}
    \begin{block}<1->{General-Purpose is suboptimal by definition}
	    All float formats will either have too much/few dynamic range or precision.
    \end{block}
    \begin{block}<2->{Hardware Reconfigurability}
	    Adapt the circuit and number representation for the corresponding task.
    \end{block}
    \begin{block}<3->{More drastically}
	    Taepout a chip for a task ?
    \end{block}
    \begin{block}<4->{Multi-Level Intermediate Representations}
	    At least, this paradigm will help in capturing optimizations from the problem to the transistors.
    \end{block}


\end{frame}


\begin{frame}
\frametitle{Publications Related to this Presentation}
\scriptsize
\begin{myitemize}
	\item \textbf{L. Ledoux} and M. Casas, “A Generator of Numerically-Tailored and High-Throughput Accelerators for Batched GEMMs,” in 2022 IEEE 30th Annual International Symposium on Field-Programmable Custom Computing Machines (FCCM), May 2022, pp. 1–10. \textit{\url{doi: 10.1109/FCCM53951.2022.9786164}}
	\item \textbf{L. Ledoux} and M. Casas, “An Open-Source Framework for Efficient Numerically-Tailored Computations,” in 2023 33rd International Conference on Field-Programmable Logic and Applications (FPL), Sep. 2023, pp. 19–26. \textit{\url{doi: 10.1109/FPL60245.2023.00011}}
	\item \textbf{L. Ledoux} and M. Casas, “The Grafted Superset Approach: Bridging Python to Silicon with Asynchronous Compilation and Beyond” in 2024 4th Workshop on Open-Source Design Automation (OSDA), hosted at DATE, March 25, 2024, at Palacio De Congresos Valencia, Spain. \textit{Available online soon.}
	\item \textbf{L. Ledoux} and M. Casas, “LLMMMM: Large Language Models Matrix-Matrix Multiplications Characterization on Open Silicon” in 2024 11th BSCSymposium, May 2024, \textit{Available online soon.}
	\item \textbf{L. Ledoux} and M. Casas, “Open-Source GEMM Hardware Kernels Generator: Toward Numerically-Tailored Computations” in 2023 10th BSCSymposium, May 2023, \textit{Available: \url{https://arxiv.org/abs/2305.18328}}
	\item \textbf{L. Ledoux} and M. Casas, “Accelerating DL inference with (Open)CAPI and posit numbers,” in OpenPOWER summit 2019, Lyon, France: linux foundation, Oct. 2019. \textit{Available: \url{https://hal.science/hal-04094850}}

\end{myitemize}
\normalsize
\end{frame}

%\begin{frame}
%
%\frametitle{Dissemination: open source code, reproducibility}
%\scriptsize
%\begin{myitemize}
%	\item \textbf{OSFNTC:} Discussed in Chapters~\ref{chapter:generator} and~\ref{chapter5:framework}, this repository contains the implementation of the Open-Source Framework for Numerically-Tailored Computations.
%		\textit{Available:\url{https://github.com/Bynaryman/OSFNTC}}
%
%	\item \textbf{Teras and Wrapped\_teras:} Presented in Chapter~\ref{chapter:generator}, these contain the code and GDS files of the tapedout chips of a Systolic Array with posits<8,0> with quire on the shuttle mpw by skywater efabless and google in 130nm.
%		\textit{Available:\url{https://github.com/Bynaryman/teras}, and \url{https://github.com/Bynaryman/wrapped\_teras}}.
%
%    \item \textbf{POF (Posit Operator Framework):} Introduced in Chapter~\ref{chapter:posit}, POF offers a comprehensive suite for developing and evaluating posit-based arithmetic operations as well as testbenches for evaluation of correctness.
%	    \textit{Available: \url{https://github.com/Bynaryman/POF}}
%
%    \item \textbf{SUF (Superset Framework):} Explored in Chapter~\ref{chapter:division}, SUF extends the capabilities of traditional ASIC flow by enhancing the parallelism in an asynchronous fashion.
%	    \textit{Available: \url{https://github.com/Bynaryman/SUF}}
%
%    \item \textbf{3lnn (Three-Layer Neural Network):} Detailed in Chapter~\ref{chapter:posit}, this repository contains the evaluated neural network topology and the pre-trained weights in posit bitstrings.
%	    \textit{Available: \url{https://github.com/Bynaryman/mnist-3lnn}}
%
%    \item \textbf{Flopoco / Virtex FPGA family:} In Chapters~\ref{chapter:posit},~\ref{chapter:generator}, and~\ref{chapter5:framework} we deffer the pipelining task of our arithmetic operators to FloPoCo. The model of our FPGA is to be found here:\url{https://gitlab.com/flopoco/flopoco/-/blob/master/code/HighLevelArithmetic/src/Targets/VirtexUltrascalePlus.cpp}, but more specially with commit hash: \url{https://gitlab.com/flopoco/flopoco/-/commit/0e0db94d2c5dc1084d477b9cb9b8b34d1a23a9e1}.
%
%    \item \textbf{VH2V:} Featured in Chapters with ASIC flow, VH2V is used as a VHDL to Verilog translator optimized for arithmetic properties, tailored for FloPoCo outputs.
%	    \textit{Available: \url{https://github.com/Bynaryman/vh2v}}
%\end{myitemize}
%\normalsize
%\end{frame}

%\begin{frame}
%\frametitle{Open Source Thesis: Reproducibility and Collaborations}
%\scriptsize
%\begin{myitemize}
%    \item \textbf{OSFNTC:} Implementation of the Open-Source Framework for Numerically-Tailored Computations. \textit{Available: \url{https://github.com/Bynaryman/OSFNTC}}
%
%    \item \textbf{Teras and Wrapped\_teras:} Code and GDS files for the Systolic Array with posits<8,0> on the SkyWater shuttle. \textit{Available: \url{https://github.com/Bynaryman/teras}, \url{https://github.com/Bynaryman/wrapped\_teras}}.
%
%    \item \textbf{POF (Posit Operator Framework):} Suite for developing and testing posit-based arithmetic operations. \textit{Available: \url{https://github.com/Bynaryman/POF}}
%
%    \item \textbf{SUF (Superset Framework):} Enhances ASIC flow by adding asynchronous parallelism. \textit{Available: \url{https://github.com/Bynaryman/SUF}}
%
%    \item \textbf{3lnn (Three-Layer Neural Network):} Neural network topology and pre-trained weights in posit format. \textit{Available: \url{https://github.com/Bynaryman/mnist-3lnn}}
%
%    \item \textbf{Flopoco / Virtex FPGA family:} Pipeline tool for arithmetic operators used in FPGA design. \textit{Commit: \url{https://gitlab.com/flopoco/flopoco/-/commit/0e0db94d2c5dc1084d477b9cb9b8b34d1a23a9e1}}
%
%    \item \textbf{VH2V:} VHDL to Verilog translator optimized for arithmetic operations. \textit{Available: \url{https://github.com/Bynaryman/vh2v}}
%\end{myitemize}
%\normalsize
%\end{frame}

\begin{frame}
\frametitle{Open Source Thesis: Reproducibility and Collaborations}
\scriptsize
\begin{columns} % Use the columns environment to place text and image side by side

\begin{column}{0.75\textwidth} % Adjust width to fit your content needs
\begin{itemize}
    \item \textbf{OSFNTC:} Implementation of the Open-Source Framework for Numerically-Tailored Computations. \textit{Available: \url{https://github.com/Bynaryman/OSFNTC}}
    \item \textbf{Teras and Wrapped\_teras:} Code and GDS files for the Systolic Array with posits<8,0> on the SkyWater shuttle. \textit{Available: \url{https://github.com/Bynaryman/teras}, \url{https://github.com/Bynaryman/wrapped\_teras}}.
    \item \textbf{POF (Posit Operator Framework):} Suite for developing and testing posit-based arithmetic operations. \textit{Available: \url{https://github.com/Bynaryman/POF}}
    \item \textbf{SUF (Superset Framework):} Enhances ASIC flow by adding asynchronous parallelism. \textit{Available: \url{https://github.com/Bynaryman/SUF}}
    \item \textbf{3lnn (Three-Layer Neural Network):} Neural network topology and pre-trained weights in posit format. \textit{Available: \url{https://github.com/Bynaryman/mnist-3lnn}}
    \item \textbf{Flopoco / Virtex FPGA family:} Pipeline tool for arithmetic operators used in FPGA design. \textit{Commit: \url{https://gitlab.com/flopoco/flopoco/-/commit/0e0db94d2c5dc1084d477b9cb9b8b34d1a23a9e1}}
    \item \textbf{VH2V:} VHDL to Verilog translator optimized for arithmetic operations. \textit{Available: \url{https://github.com/Bynaryman/vh2v}}
\end{itemize}
\end{column}

\begin{column}{0.25\textwidth}
%\begin{figure}[H]
\centering
\includegraphics[width=\columnwidth]{qart.png}
\caption{Is that Ada indicating the repositories' URL?}
%\end{figure}
\end{column}

\end{columns}
\normalsize
\end{frame}

\begin{frame}[noframenumbering,plain]
\begin{center}
\begin{figure}[H]
	\centering
	%\animategraphics[autoplay,loop,width=0.3\textwidth]{12}{asap-}{0}{49}
	\animategraphics[autoplay,palindrome,width=0.3\textwidth]{12}{asap-}{0}{49}
\end{figure}

\Huge \textcolor{cvut_blue}{Thank You}
\end{center}
\end{frame}
