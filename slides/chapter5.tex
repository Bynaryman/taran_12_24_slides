\graphicspath{{../../../PhD/paper_factory/thesis_louis/Chapter5/Figs/}}
%\subsection{Executive Summary}
%\begin{frame}
%    \frametitle{Current Progress}
%
%    % Adding a thematic title for the slide that fits with the chapter
%    \centering
%    \huge ``An Open-Source Framework for Efficient Numerically-Tailored Computations''
%    \normalsize
%
%    \vspace{1em} % Add some vertical space before the table of contents
%
%    % Only shows the current section and its subsections, hides everything else
%    \tableofcontents[currentsection,
%                     subsectionstyle=show/shaded/hide,
%                     sectionstyle=show/hide]
%
%    % Explanation of options:
%    % - currentsection: Highlights the current section
%    % - subsectionstyle=show/show/hide: The current subsection and its siblings in the current section are shown
%    % - sectionstyle=show/hide: Shows only the current section, hides other sections
%\end{frame}
%
%
%\begin{frame}
%    \frametitle{Executive Summary}
%
%    % Uncover each block one by one
%        \begin{block}<1->{GEMM Multiplication Kernel}
%		A large variety of scientific applications rely on \textbf{GEMM } kernel. The \textbf{numerical requirements are very heterogeneous}, e.g., AI or weather forecasting.
%        \end{block}
%
%        \begin{exampleblock}<2->{Lack of Hardware Precision Adaptability}
%		\textbf{Extended} or adaptative \textbf{software precision is expensive} and often the computations land in commodity hardware.
%        \end{exampleblock}
%
%        \begin{alertblock}<3->{Our Proposal}
%		We propose an \textbf{Open-Source framework} to bridge high-level libraries to numerically aware and accelerated GEMM kernels.
%        \end{alertblock}
%\end{frame}

\subsection{Bridging to real workloads}
\subsubsection{Open Source SW/HW Co-designed Framework}
\begin{frame}
    \frametitle{Current Progress}

    % Only shows the current section and its subsections, hides everything else
    \tableofcontents[currentsection,
                     subsectionstyle=show/shaded/hide,
                     sectionstyle=show/hide]

    % Explanation of options:
    % - currentsection: Highlights the current section
    % - subsectionstyle=show/show/hide: The current subsection and its siblings in the current section are shown
    % - sectionstyle=show/hide: Shows only the current section, hides other sections
\end{frame}

\begin{frame}
    \frametitle{Open Source SW/HW Co-designed Framework}
    \begin{figure}
      \centering
      \includegraphics[width=\textwidth]{overall_framework.pdf}
      \caption{The 2 phases of the framework: right, the a priori Hardware generation flow, and left, the runtime execution flow.}
    \end{figure}

\end{frame}


\begin{frame}
    \frametitle{Framework: Hardware Side}

    \begin{columns}

    \begin{column}{0.5\textwidth} % Adjust width as needed
        \begin{itemize} % Incremental overlay specifications
	    \item<1-> Configuration files to bitstream
	    \item<1-> Modified FloPoCo~\cite{flopoco1}
            \item<1-> OpenCAPI offers $\approx$23GBp/s with 1024-bit bus @200MHz
        \end{itemize}
    \end{column}

    \begin{column}{0.5\textwidth}
        \begin{figure}
            \centering
            \includegraphics[width=\textwidth]{framework_HW.pdf}
            \caption{Automated HW generation}
        \end{figure}
    \end{column}

    \end{columns}

\onlyfootcite{flopoco1}

\end{frame}

\begin{frame}
    \frametitle{Framework: Software Side}

    \begin{columns}

    \begin{column}{0.35\textwidth} % Adjust width as needed
	\only<1>{
	Strategy:
        \begin{itemize}
		\item<1-> HPC codes rely on \textbf{BLAS calls}
		\item<1-> \textbf{Numpy, Pytorch}
		\item<1-> Intercept, cast, dispatch
        \end{itemize}
	}
	\only<2>{
	\begin{itemize}

        	% User Process items
        	\item \textbf{User Process:}
        	\begin{enumerate}
        	    \item Matrix Allocation
        	    \item Call \texttt{gemm()}
        	\end{enumerate}

        	% Spacer
        	\vspace{1em}

        	% Our Library items
        	\item \textbf{Our lib:}
        	\begin{enumerate}
		    \setcounter{enumi}{2}
        	    \item Arithmetic cast
        	    \item Signal DMA properties
        	    \item HW address translation
        	    \item DMA starts
        	\end{enumerate}
    \end{itemize}
    }
    \end{column}

    \begin{column}{0.65\textwidth}
        \begin{figure}
            \centering
            \includegraphics[width=\textwidth]{framework_SW.pdf}
            \caption{Software flow}
        \end{figure}
    \end{column}

    \end{columns}

\end{frame}

\begin{frame}
	\frametitle{SW Code Example}

	\small
    \begin{figure}
        \centering
        \includegraphics[width=0.8\textwidth]{listing1.png}
	    \vspace{-3mm}
        \caption{Calling twice the same user application (python) with our BLAS}
    \end{figure}
	    \vspace{-2mm}

    \begin{figure}
        \centering
        \includegraphics[width=0.8\textwidth]{listing2.png}
	    \vspace{-3mm}
        \caption{gemm.py: a python program calling GEMMs. No changes in the code.}
    \end{figure}
	\normalsize

\end{frame}

\subsubsection{Workload Evaluation}
%\begin{frame}
%    \frametitle{Current Progress}
%
%    % Only shows the current section and its subsections, hides everything else
%    \tableofcontents[currentsection,
%                     subsectionstyle=show/shaded/hide,
%                     sectionstyle=show/hide]
%
%    % Explanation of options:
%    % - currentsection: Highlights the current section
%    % - subsectionstyle=show/show/hide: The current subsection and its siblings in the current section are shown
%    % - sectionstyle=show/hide: Shows only the current section, hides other sections
%\end{frame}

\begin{frame}
	\frametitle{Evaluation of HPC Codes: Sea Surface Height (SSH)}

%\scriptsize

\begin{columns}

\begin{column}{0.5\textwidth}
    \begin{myitemize}
        \item<1-> Ocean circulation models
	\item<1-> Monitors climate changes~\cite{barrick_ocean}
	\item<1-> \textbf{Precision-hungry} workload
	\item<1-> Alternating signs and spans over \textbf{15 orders of magnitude}
	\item<1-> Can be solved without hardware
	\item<1-> Kahan SCS, DCS~\cite{kahan_pracniques_1965}
	\item<1-> Extended Software Precision~\cite{bailey_high-precision_2009})
    \end{myitemize}
\end{column}
\begin{column}{0.5\textwidth}
    \begin{figure}
	    \vspace{2mm}
        \includegraphics[width=\textwidth]{4_forloops.pdf}
	    \vspace{-11mm}
	    \caption{Double-precision errors.}
    \end{figure}
\end{column}

\end{columns}


\onlyfootcite{barrick_ocean}
\onlyfootcite{kahan_pracniques_1965}
\onlyfootcite{bailey_high-precision_2009}
\normalsize

	\note[item]  {(conv, im2col, FC)}

\end{frame}

\begin{frame}
\frametitle{Evaluation of HPC Codes: Artificial Intelligence}

%\scriptsize

\begin{columns}

\begin{column}{0.5\textwidth}
    %\textbf{Artificial Intelligence (AI)}
    \begin{myitemize}
	\item GEMMs (im2col, FC, Conv)
        %\item Used to solve many modern problems (image recognition, NLP)
	\item \textbf{Precision-resilient} workloads~\cite{johnson_rethinking_2018}
        \item Accumulations of closely related values~\cite{li_robustness-aware_2021}
    \end{myitemize}
    %\vspace{-2mm} % Adjust spacing as needed
\end{column}
\begin{column}{0.5\textwidth}
    \begin{figure}
        \includegraphics[width=\textwidth]{histograms.jpg}
	    \caption{Weights histograms Alexnet/Resnet18}
    \end{figure}
\end{column}
\end{columns}


\onlyfootcite{li_robustness-aware_2021}
\onlyfootcite{johnson_rethinking_2018}
\normalsize

	\note[item]  {(conv, im2col, FC)}

\end{frame}

\begin{frame}
\frametitle{Evaluation: SSH results}

\begin{columns}

\begin{column}{0.5\textwidth}
	\tiny
%\scriptsize
\begin{myitemize}
	\item<1-> \textbf{IEEE754-64-FDP} vs. \textbf{IEEE754-64/128 FMAs}
	\tiny
    \begin{myitemize}
	\tiny
        \item 64-bit format + 91-bit FDP: <ovf:30,msb:30,lsb:-30>
	\item 128-bit quad-precision FMA
        \item 64-bit double-precision FMA
    \end{myitemize}
	\item<2-> \textbf{10 shuffles} per chunk size
	\item<2-> \textbf{4 metrics}: mean, RSD, Accuracy, Accuracy cost
    \tiny
    \item<3-> The FDP maintains \textbf{reproducibility} unlike the FMA
    \item<3-> The FDP always exhibits \textbf{52} correct bits
    \begin{myitemize}
	\tiny
	\item \textbf{5x} better than quad-precision
	\item \textbf{27.7x} better than double-precision
    \end{myitemize}

	\tiny
    \item<4-> At 200MHz, power consumptions are:
    \begin{myitemize}
	\tiny
        \item 0.266W for IEEE754-64
        \item 0.549W for IEEE754-128
        \item 0.491W for 91-bit FDP
    \end{myitemize}
	\tiny
    \item<5-> For the same Wattage, our FDP generates:
    \begin{myitemize}
	\tiny
	\item \textbf{5.6x} more correct bits than IEEE754-128
	\item \textbf{15.1x} more correct bits than IEEE754-64
    \end{myitemize}
\end{myitemize}
\end{column}

\tiny
\begin{column}{0.5\textwidth}
\centering
\includegraphics[width=0.85\columnwidth]{SSH.pdf}
\caption{IEEE754-64/128 FMAs vs. [64+91]-FDP}
\end{column}

\end{columns}
\normalsize
\end{frame}

\begin{frame}
\frametitle{Evaluation: AI results}

\begin{columns}

\begin{column}{0.4\textwidth}
\tiny
\begin{itemize}
	\item<1-> \textbf{Pytorch code without any changes}
	\item<1-> Top1/5 scores and \textbf{``score costs''}
	\vspace{3mm}
    \item<2-> Evaluated models:
        \begin{itemize}
\tiny
            \item ResNet18
            \item ResNet34
            \item ResNet50
            \item DenseNet121
            \item DenseNet169
            \item VGG11\_BN
        \end{itemize}
    \item<3-> Evaluated datasets:
        \begin{itemize}
\tiny
            \item CIFAR10
            \item ImageNet
        \end{itemize}
    \item<4-> Evaluated computer formats:
        \begin{itemize}
\tiny
            \item IEEE754-32
            \item BrainFloat16
        \end{itemize}
    \item<5-> 27 accumulators and conventional FMA
    \item<5-> A total of \textbf{384} combinations
\end{itemize}
\end{column}

\begin{column}{0.6\textwidth}
\tiny
\centering
\includegraphics[width=\columnwidth]{nn_accuracy_and_acc_per_power_all_subplot.png}
\caption{Top1/5 scores/costs across different models/formats/datasets/accumulators}
\end{column}

\end{columns}
\normalsize
\end{frame}

\begin{frame}
\frametitle{A zoom on the LSB case}
\begin{figure}
	\centering
	\includegraphics[width=0.75\textwidth]{zoom_accuracy.png}
	\vspace{-10pt}
	\caption{The impact of lowering number of LSBs on Top1/5 scores and score costs.}
\end{figure}
\end{frame}

\begin{frame}
\frametitle{A zoom on the LSB + densenet121 + ImageNet case}
\begin{figure}
	\centering
	\includegraphics[width=0.75\textwidth]{zoom_accuracy_highlight.png}
	\vspace{-10pt}
	\caption{The impact of lowering number of LSBs on Top1/5 scores and score costs.}
\end{figure}
\end{frame}

\begin{frame}
\frametitle{Evaluation: AI results}

\begin{figure}
	\centering
	\includegraphics[width=0.97\columnwidth]{nn_power_all_subplots.pdf}
	\vspace{-4mm}
	\caption{Top1 scores VS. Energy cost of inferencing validation set}
\end{figure}

\normalsize
\end{frame}

\begin{frame}
    \frametitle{Zoom on ImageNet + Resnet50}

	\only<1-4>{
	\begin{figure}[H]
            \centering
		%\vspace{1mm}
	    	\only<1>{
	            	\includegraphics[width=\textwidth]{zoom_imagenet_resnet50.png}
			%\vspace{-2mm}
			\caption{Test Set Inference Cost VS. Top1 Score.}
		}
	    	\only<2>{
	            	\includegraphics[width=\textwidth]{zoom_imagenet_resnet50_2.png}
			%\vspace{-2mm}
			\caption{Test Set Inference Cost VS. Top1 Score.}
		}
	    	\only<3>{
	            	\includegraphics[width=\textwidth]{zoom_imagenet_resnet50_3.png}
			%\vspace{-2mm}
			\caption{Test Set Inference Cost VS. Top1 Score.}
		}
	    	\only<4>{
	            	\includegraphics[width=\textwidth]{zoom_imagenet_resnet50_4.png}
			%\vspace{-2mm}
			\caption{Test Set Inference Cost VS. Top1 Score.}
		}
	\end{figure}
	}
%            \item goal: 84\% top 1 score
%            \item FDP<9,6-20> saves 25Wh for bf16 and 100Wh for FP32


	\only<5>{
        \begin{block}{Graphical Superiority}
		Graphically, it always exists one \textbf{FDP} to the left (indicating lower energy) and are never below (ensuring equal or greater accuracy) conventional FMAs, \textbf{highlighting their superior performance}
        \end{block}
	}
    \normalsize

\end{frame}

\subsection{Conclusions}
%\begin{frame}
%    \frametitle{Current Progress}
%
%    % Only shows the current section and its subsections, hides everything else
%    \tableofcontents[currentsection,
%                     subsectionstyle=show/shaded/hide,
%                     sectionstyle=show/hide]
%
%    % Explanation of options:
%    % - currentsection: Highlights the current section
%    % - subsectionstyle=show/show/hide: The current subsection and its siblings in the current section are shown
%    % - sectionstyle=show/hide: Shows only the current section, hides other sections
%\end{frame}

\begin{frame}
    \frametitle{Summary of Insights and Key Contributions}

	\begin{alertblock}<1->{Open-Source Framework}
		A \textbf{bridge} from user applications to silicon, focusing on \textbf{tailored arithmetic} solutions.
	\end{alertblock}
	\vspace{-2mm}
	\begin{alertblock}<2->{Transparent Integration}
		Seamlessly integrates \textbf{without code modification}, agnostic of underlying hardware.
	\end{alertblock}
	\vspace{-2mm}
	\begin{block}<3->{Enhancing State-of-the-Art}
		Improves \textbf{energy efficiency and accuracy} across \textbf{drastically diverse HPC applications}.
	\end{block}
	\vspace{-2mm}
	\begin{exampleblock}<4->{Findings}
		System-wise Arithmetic and numerically-tailored solutions alleviate the power wall (A), and the concern of workload sparsity (B).
	\end{exampleblock}
\end{frame}
