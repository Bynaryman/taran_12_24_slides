\usepackage[utf8]{inputenc}

\usepackage{comment}
\usetheme{Madrid}
\useoutertheme{miniframes}
%\useoutertheme{infolines}

\setbeamertemplate{miniframes}{%
  \vspace{-2pt}% Reduce vertical space above the dots
  \insertnavigation{\textwidth}%
}

\setbeamertemplate{headline}{%
  \leavevmode%
  \hbox{%
    \begin{beamercolorbox}[wd=\paperwidth,ht=2.25ex,dp=6.5ex]{section in head/foot}
      \insertnavigation{\paperwidth}
    \end{beamercolorbox}%
  }
}


\usefonttheme{professionalfonts}
%\usefonttheme{serif}

%\setbeamercovered{transparent}
\setbeamercovered{dynamic}

\definecolor{cvut_navy}{HTML}{0065BD}
\definecolor{cvut_blue}{HTML}{6AADE4}
\definecolor{cvut_gray}{HTML}{156570}
\usepackage[backend=bibtex,style=trad-alpha,maxcitenames=3,doi=false,isbn=false,url=false]{biblatex} %for footcite, mycite
%\usepackage[backend=bibtex,maxbibnames=99,style=authoryear, natbib=true,autocite=footnote]{biblatex}
\addbibresource{references.bib}

\makeatletter
\newrobustcmd{\mkbibblfootnote}[1]{%
  \iftoggle{blx@footnote}
    {\blx@warning{Nested notes}%
     \addspace\mkbibparens{#1}}
    {\unspace
     \ifnum\blx@notetype=\tw@
       \expandafter\@firstoftwo
     \else
       \expandafter\@secondoftwo
     \fi
       {\csuse{blx@theendnote}{\protecting{\blxmkbibnote{end}{#1}}}}
       {\csuse{blfootnote}{\protecting{\blxmkbibnote{foot}{#1}}}}}}
\newcommand\blfootnote[1]{\begingroup\let\thefootnote\relax\footnotetext{#1}\endgroup}
\makeatother

\DeclareCiteCommand{\mycite}[\mkbibbrackets]
  {\usebibmacro{prenote}}%text before the cite. e.g.: (see [XX])
  %{\usedriver
  %   {\DeclareNameAlias{sortname}{default}}
  %   {\thefield{entrytype}}}
  %{\usebibmacro{citeindex}\usebibmacro{cite}}%as command \cite
  {%
		%{\bibopenbracket\usebibmacro{citeindex}\usebibmacro{cite}\bibclosebracket}%
		{\usebibmacro{citeindex}\usebibmacro{cite}}%
		%\textsuperscript{\bibopenbracket\usebibmacro{citeindex}\usebibmacro{cite}\bibclosebracket}%
		\edef\body{\usebibmacro{cite}\usedriver{\DeclareNameAlias{sortname}{default}}{\thefield{entrytype}}}%
		%\ifciteseen{}{}
		\mkbibblfootnote{%
			\mkbibbrackets{\usebibmacro{cite}}%
			\setunit{\addspace}%
			%\usedriver{\DeclareNameAlias{sortname}{default}}{\thefield{entrytype}}%
			{\usebeamercolor[fg]{bibliography entry author}\printnames{labelname}}%
			\setunit{\nametitledelim}%
			{\usebeamercolor[fg]{bibliography entry title}\printfield[citetitle]{labeltitle}}%
			\setunit{\addspace}%
			%{\usebeamercolor[fg]{bibliography entry note}\printfield{series}}%
			{\usebeamercolor[fg]{bibliography entry note}\printfield[journaltitle]{shortjournal}}%
			%{\usebeamercolor[fg]{bibliography entry note}\printfield[journaltitle]{journaltitle}}%
			%{\usebeamercolor[fg]{bibliography entry note}\printfield[journaltitle]{journalsubtitle}}%
			\setunit{\addspace}%
			{\usebeamercolor[fg]{bibliography entry note}\printfield[parens]{year}}%
			%\printfield[parens]{year}%
			%bibliography entry title
			%bibliography entry journal
			%bibliography entry note
			%bibliography entry title
			%\printnames{labelname}%
			%\setunit{\addspace}%
			%\printfield[parens]{year}%
			%\setunit{\nametitledelim}%
			%\printfield[citetitle]{labeltitle}%
			}%
  }
  %text of the cite
  {\multicitedelim}%the separator between several cites of the same command
  {\usebibmacro{postnote}}%text after the cite
%\renewcommand\mycite{\cite}
\DeclareCiteCommand{\onlyfootcite}
  {\usebibmacro{prenote}}%text before the cite. e.g.: (see [XX])
  {%
		\mkbibblfootnote{%
			\mkbibbrackets{\usebibmacro{cite}}%
			\setunit{\addspace}%
			{\usebeamercolor[fg]{bibliography entry author}\printnames{labelname}}%
			\setunit{\nametitledelim}%
			{\usebeamercolor[fg]{bibliography entry title}\printfield[citetitle]{labeltitle}}%
			\setunit{\addspace}%
			{\usebeamercolor[fg]{bibliography entry note}\printfield[journaltitle]{shortjournal}}%
			\setunit{\addspace}%
			{\usebeamercolor[fg]{bibliography entry note}\printfield[parens]{year}}%
			}%
  }
  %text of the cite
  {\multicitedelim}%the separator between several cites of the same command
  {\usebibmacro{postnote}}%text after the cite
\renewcommand\multicitedelim{\addcomma\space}


\setbeamercolor{section in toc}{fg=black,bg=yellow}
\setbeamercolor{alerted text}{fg=cvut_blue}
\usepackage{tikzsymbols}
\usepackage{textcomp}
\usepackage{parskip}
\definecolor{darkblue}{rgb}{0, 0, 0.5}
\definecolor{babyblue}{rgb}{0.54, 0.81, 0.94}
\usepackage{pgf}
\usepackage{color,soul}
\usepackage{tcolorbox}
\tcbuselibrary{skins}
\usepackage{hyperref}
\usepackage{xcolor,soul}
\definecolor{lightblue}{rgb}{.90,.95,1}
\sethlcolor{lightblue}
\renewcommand<>{\hl}[1]{\only#2{\beameroriginal{\hl}}{#1}}
\setbeamertemplate{page number in head/foot}[framenumber]

% fix cite in columns
\addtobeamertemplate{footnote}{\vspace{-6pt}\advance\hsize-0.5cm}{\vspace{6pt}}
\makeatletter
% Alternative A: footnote rule
\renewcommand*{\footnoterule}{\kern -3pt \hrule \@width 2in \kern 8.6pt}
% Alternative B: no footnote rule
% \renewcommand*{\footnoterule}{\kern 6pt}
\makeatother

\definecolor{babyblue}{rgb}{0.54, 0.81, 0.94}
\definecolor{babypink}{rgb}{0.96, 0.76, 0.76}
\definecolor{blue(ncs)}{rgb}{0.0, 0.53, 0.74}
\definecolor{pistachio}{rgb}{0.58, 0.77, 0.45}
\definecolor{darksalmon}{rgb}{0.91, 0.59, 0.48}
\definecolor{lightsalmonpink}{rgb}{1.0, 0.6, 0.6}
\definecolor{columbiablue}{rgb}{0.61, 0.87, 1.0}
\definecolor{corn}{rgb}{0.98, 0.93, 0.36}
\definecolor{jonquil}{rgb}{0.98, 0.85, 0.37}
\definecolor{bananayellow}{rgb}{1.0, 0.88, 0.21}


\usepackage{courier}



\makeatletter
\newcommand\SoulColor{%
  \let\set@color\beamerorig@set@color
  \let\reset@color\beamerorig@reset@color}
\makeatother
\SoulColor
\usepackage{amsmath, bm}
\usepackage{tikz}
\newcommand{\highlightt}[1]{%
  \colorbox{blue!40}{$\displaystyle#1$}}


\newenvironment<>{problock}[1]{%
  \begin{actionenv}#2%
      \def\insertblocktitle{#1}%
      \par%
      \mode<presentation>{%
       % \setbeamercolor{block title}{fg=white,bg=orange!20!black}
        %\setbeamercolor{block title}{fg=white,bg=red!10!black}
        \setbeamercolor{block title}{fg=white,bg=cvut_blue}
       \setbeamercolor{block body}{fg=black,bg=white!50}
       \setbeamercolor{itemize item}{fg=orange!20!black}
       \setbeamertemplate{itemize item}[triangle]
     }%
      \usebeamertemplate{block begin}
    \par\usebeamertemplate{block end}
    \end{actionenv}
    }

    \newcommand<>{\uncovergraphics}[2][{}]{
    % Taken from: <https://tex.stackexchange.com/a/354033/95423>
    \begin{tikzpicture}
    \node[anchor=south west,inner sep=0] (B) at (4,0)
        {\includegraphics[#1]{#2}};
    \alt#3{}{%
        \fill [draw=none, fill=background, fill opacity=0.9] (B.north west) -- (B.north east) -- (B.south east) -- (B.south west) -- (B.north west) -- cycle;
    }
    \end{tikzpicture}
}




\newlength\dlf
\newcommand\alignedbox[3][yellow]{
  % #1 = color (optional, defaults to yellow)
  % #2 = before alignment
  % #3 = after alignment
  &
  \begingroup
  \settowidth\dlf{$\displaystyle #2$}
  \addtolength\dlf{\fboxsep+\fboxrule}
  \hspace{-\dlf}
  \fcolorbox{red}{#1}{$\displaystyle #2 #3$}
  \endgroup
}

\usepackage{collcell}
\usepackage{booktabs}
\usepackage{etoolbox}
%\usepackage{remreset}% tiny package containing just the \@removefromreset command
\makeatletter

\usepackage{xcoffins}
\NewCoffin\tablecoffin
\NewDocumentCommand\Vcentre{m}
  {%
    \SetHorizontalCoffin\tablecoffin{#1}%
    \TypesetCoffin\tablecoffin[l,vc]%
  }



\usepackage{pgfpages}

%% Important
%%% show note or disable note.
%\setbeameroption{show notes on second screen=right} % Both

\setbeamertemplate{note page}{\pagecolor{gray!5}\insertnote}\usepackage{palatino}



\usepackage{xcolor}
\usepackage{soul}

\usepackage{etoolbox}
\makeatletter
%\patchcmd{\slideentry}{\ifnum#2>0}{\ifnum2>0}{}{\@error{unable to patch}}% replace the subsection number test with a test that always returns true
\makeatother

\setbeamercolor*{palette primary}{bg=cvut_navy,fg=gray!20!white}
\setbeamercolor*{palette secondary}{bg=cvut_navy,fg=gray!20!white} % no color
%\setbeamercolor*{palette secondary}{bg=cvut_navy,fg=cvut_navy}
%\setbeamercolor*{palette secondary}{bg=cvut_blue,fg=white}
\setbeamercolor*{palette tertiary}{parent=palette primary} % color of the top and date
\setbeamercolor*{palette quaternary}{fg=cvut_navy,bg=gray!5!white}
\setbeamercolor*{sidebar}{fg=cvut_navy,bg=gray!15!white}
\usepackage[first=0,last=9]{lcg}
\newcommand{\ra}{\rand0.\arabic{rand}}
\usepackage{color, colortbl}
\usepackage{stackengine,tikz}
\usepackage{transparent}
\usepackage{pgfpages}
%\usepackage{graphicx}% http://ctan.org/pkg/graphicx
\usepackage{booktabs}% http://ctan.org/pkg/booktabs
%\setbeameroption{show notes}
\colorlet{Gray}{gray!30}



%\setbeameroption{show notes on second screen=right}
\setbeamercolor{titlelike}{parent=palette primary}
\setbeamercolor{frametitle}{parent=palette primary}

\setbeamercolor{B}{bg=red!30,fg=black}


\setbeamertemplate{section in toc}[default]
\setbeamercolor{itemize item }{fg=blue}
\setbeamertemplate{itemize item}[circle]

\setbeamercolor*{separation line}{}
\setbeamercolor*{fine separation line}{}

\setbeamertemplate{navigation symbols}{}
\setbeamertemplate{caption}{\raggedright\insertcaption\par}

%\setbeamercolor*{block title example}{fg=blue!50,bg= blue!10}

%\setbeamercolor*{block title example}{fg=white,bg= cvut_navy}
\setbeamercolor*{block title example}{fg=white,bg=purple!75!black}
\setbeamercolor*{block body example}{fg= black, bg= white}


\setbeamercolor{itemize item}{fg=cvut_navy} % all frames will have red bullets
%\setbeamercolor{block title}{bg=red!30,fg=black}
\setbeamertemplate{subsection in toc}[subsections numbered]


\makeatother


\usepackage{eqnarray,amsmath}
\usepackage{amsfonts}
\usepackage{amssymb}
\usepackage{graphicx}
\usepackage{animate}
\usepackage{booktabs}
\usepackage{bm}
\usepackage{mathtools}
\usepackage[T1]{fontenc}
\usepackage{lmodern}
\usepackage{booktabs}
\usepackage{threeparttable}
%\usepackage[inline]{enumitem} % for enum with more control
%\setbeamertemplate{enumerate item}{(\alph{enumi})}
%\setbeamertemplate{enumerate subitem}{(\roman{enumii})}
% Define new itemize environment
%\newenvironment{myitemize}
%{   \begin{itemize}
%    \renewcommand\labelitemi{} % Removes the bullet
%    \setlength{\itemsep}{0pt} % Remove space between items
%    \setlength{\parskip}{0pt} % Remove space between paragraphs within an item
%    \setlength{\parsep}{0pt} % Remove paragraph separation within an item
%}
%{   \end{itemize}   }
\newenvironment{myitemize}
{   \begin{itemize}
    \renewcommand\labelitemi{} % Removes the bullet at the first level
    \renewcommand\labelitemii{} % Removes the bullet at the second level
    \renewcommand\labelitemiii{} % Removes the bullet at the third level
    \renewcommand\labelitemiv{} % Removes the bullet at the fourth level
    \setlength{\itemsep}{0pt} % Remove space between items
    \setlength{\parskip}{0pt} % Remove space between paragraphs within an item
    \setlength{\parsep}{0pt} % Remove paragraph separation within an item
}
{   \end{itemize}   }

\newenvironment{myenumerate}
{   \begin{enumerate}
    \renewcommand\labelitemi{} % Removes the bullet at the first level
    \renewcommand\labelitemii{} % Removes the bullet at the second level
    \renewcommand\labelitemiii{} % Removes the bullet at the third level
    \renewcommand\labelitemiv{} % Removes the bullet at the fourth level
    \setlength{\itemsep}{0pt} % Remove space between items
    \setlength{\parskip}{0pt} % Remove space between paragraphs within an item
    \setlength{\parsep}{0pt} % Remove paragraph separation within an item
}
{   \end{enumerate}   }

\usetikzlibrary{overlay-beamer-styles}
 \tikzset{
    highlight on/.style={alt={#1{fill=red!80!black,color=red!80!black}{fill=gray!30!white,color=gray!30!white}}},
}


%\setbeamercolor{block title}{fg=darkred,bg=structure.fg!10!bg!10!bg}
%\setbeamercolor{block body}{use=block title,bg=block title.bg}
\definecolor{NormalBlue}{RGB}{200,200,255}
%\setbeamercolor{block title}{bg=NormalBlue}yellow
%\setbeamercolor{block title}{fg=black, bg=yellow}
\setbeamercolor{block2}{use=structure,fg=white,bg=purple!75!black}
% definice makra
\def\bq{\mbox{\kern.1ex\protect\raisebox{-1.3ex}[0pt][0pt]{''}\kern-.1ex}}
\def\eq{\mbox{\kern-.1ex``\kern.1ex}}
\def\ifundefined#1{\expandafter\ifx\csname#1\endcsname\relax }%
\ifundefined{uv}%
        \gdef\uv#1{\bq #1\eq}
\fi

